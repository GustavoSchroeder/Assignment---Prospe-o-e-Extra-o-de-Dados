\documentclass[conference]{IEEEtran}
\IEEEoverridecommandlockouts
% The preceding line is only needed to identify funding in the first footnote. If that is unneeded, please comment it out.
\usepackage{cite}
\usepackage{amsmath,amssymb,amsfonts}
\usepackage{algorithmic}
\usepackage{graphicx}
\usepackage{textcomp}
\usepackage{xcolor}
\def\BibTeX{{\rm B\kern-.05em{\sc i\kern-.025em b}\kern-.08em
    T\kern-.1667em\lower.7ex\hbox{E}\kern-.125emX}}
\begin{document}

\title{Conference Paper Title*\\
{\footnotesize \textsuperscript{*}Note: Sub-titles are not captured in Xplore and
should not be used}
\thanks{Identify applicable funding agency here. If none, delete this.}
}

\author{\IEEEauthorblockN{1\textsuperscript{st} Gustavo Lazarotto Schroeder}
\IEEEauthorblockA{\textit{dept. name of organization (of Aff.)} \\
\textit{Universidade do Vale do Rio dos Sinos (UNISINOS)}\\
São Leopoldo, Brazil \\
gustavo.schroeder@outlook.com}
}

\maketitle

\begin{abstract}
This document is a model and instructions for \LaTeX.
This and the IEEEtran.cls file define the components of your paper [title, text, heads, etc.]. *CRITICAL: Do Not Use Symbols, Special Characters, Footnotes, 
or Math in Paper Title or Abstract.
\end{abstract}

\begin{IEEEkeywords}
component, formatting, style, styling, insert
\end{IEEEkeywords}

\section{Introduction}
Modern technology has paved the way for the internet and devices like  smartwatches and smartphones. Computers are increasingly faster, more portable, and higher-powered than ever before. Thanks to rapid advancements in technology people are more connected than ever, smartphones provide connectivity to limitless information on-demand, enabling people to solve a wide variety of everyday problems \cite{b1}. 

This kind of connection brings a melancholy side, like internet overuse and screen addiction \cite{b2}, fear of missing out, anxiety and depression \cite{b3}, mobile game addiction relation to depression, social anxiety, and loneliness \cite{b4}, and a growing fear of being without a smartphone \cite{b5}. 

According to \cite{b6} many individuals have become addicted to using technologies and, as a consequence have experience negative mental effects, furthermore the home confinement due to the COVID-19 pandemic may have worsened this situation.

At present, there are studies that explore internet
addiction disorder (IAD) level based on the Internet login data \cite{b7}, IAD based on the browser history (Internet Addiction and Mental Health Prediction Using Ensemble Learning Based on Web Browsing History), and Smartphone Addiction based on questionnaires and data collected from the use of smartphones.

In this paper we propose the use of Information Fusion, that is the integration of data and knowledge from several sources (A Review of Data Fusion Techniques), to create a digital intoxication scale which encompasses several technologies and mental problems.

\section{Related Work}

\subsection{Analysis of behavioral characteristics of smartphone addiction using data mining}

This study analised smartphone addiction by considering the differences between smartphone usage patterns as well as cognition. Proposing a method that involves automatically collecting and analyzing data through an app. With the use of an app collection information about the smartphone use and questionnaires, they used data mining to divide user users into groups of high-risk, potential-risk and normal users. \cite{b7} identified that the variables “number of screen turns” and “actual use time—perceived use time” are more influential on poisoning than the previous research that used the smartphone as poisoning.

\subsection{Pattern Study on smartphone addiction group}

Kim \cite{b9} identified smartphone addiction trends by obtaining addiction patterns based on the degree of smartphone overuse. The study showed that smartphone addicts suffered from the phenomenon of sleep deprivation, delaying their bedtime beyond midnight when engaging in the use of the phone.

\subsection{Pattern Study on smartphone addiction group}
In their study on smartphone usage pattern, Ryu et al.\cite{b10} studied a smartphone addiction and disease prevention system through smartphone usage pattern collection and analysis. Smartphone usage pattern is defined as the current state of the smartphone and user. In their study, the usage pattern data was collected through the orientation sensor and display activation state available in the smartphone. In particular, the objective of their study was to prevent the prevalence of addiction and illness caused by smartphone overuse by educating the users about usage patterns through the collected data. However, although their proposed method can be used to predict smartphone addiction disease according to the behavior type, its simplicity in evaluating for the disease using only the smartphone usage time serves as a limitation.

\subsection{Correlation Analysis of Internet Addiction with Daily Behavior: A Data-Driven Method}

Zhang \cite{b9} used the Internet Addiction Test (IAT) and context histories to identify students who were addicted to the internet using Internet data from the Campus, thus creating the level of internet addiction: The level of internet addiction is a comprehensive and difficult to quantify, so we selected five measures that indicate their grade based on previous psychological research, that is, network traffic, total online time on weekends, total online time on work days, average initial login time and last average logout time.

\subsection{Internet Addiction and Mental Health Prediction Using Ensemble Learning Based on Web Browsing History}

This study \cite{b11} analyzed the web browser history of 30 undergraduate students at Universitas Indonesia (UI) during the course of five weeks. The data was analyzed using the support vector machine (SVM) with radial basis function (RBF) kernel as a machine learning method for prediction. The results were subsequently compared using ensemble learning, such as random forest (RF) and gradient boosting (GB). It was then matched with respondents’ responses to an Internet Addiction Test (IAT) questionnaire, which measures IAD levels. 

The extracted features became input to classify participants’ IAD. The results were compared with their IAD results from the IAT questionnaire. Machine learning was also employed to classify the input into respondents’ general health (GH) status, which was matched with their responses to the GHQ-12 questionnaire.

\section{Prepare Your Paper Before Styling}
Before you begin to format your paper, first write and save the content as a 
separate text file. Complete all content and organizational editing before 
formatting. Please note sections \ref{AA}--\ref{SCM} below for more information on 
proofreading, spelling and grammar.

Keep your text and graphic files separate until after the text has been 
formatted and styled. Do not number text heads---{\LaTeX} will do that 
for you.

\subsection{Abbreviations and Acronyms}\label{AA}
Define abbreviations and acronyms the first time they are used in the text, 
even after they have been defined in the abstract. Abbreviations such as 
IEEE, SI, MKS, CGS, ac, dc, and rms do not have to be defined. Do not use 
abbreviations in the title or heads unless they are unavoidable.

\subsection{Units}
\begin{itemize}
\item Use either SI (MKS) or CGS as primary units. (SI units are encouraged.) English units may be used as secondary units (in parentheses). An exception would be the use of English units as identifiers in trade, such as ``3.5-inch disk drive''.
\item Avoid combining SI and CGS units, such as current in amperes and magnetic field in oersteds. This often leads to confusion because equations do not balance dimensionally. If you must use mixed units, clearly state the units for each quantity that you use in an equation.
\item Do not mix complete spellings and abbreviations of units: ``Wb/m\textsuperscript{2}'' or ``webers per square meter'', not ``webers/m\textsuperscript{2}''. Spell out units when they appear in text: ``. . . a few henries'', not ``. . . a few H''.
\item Use a zero before decimal points: ``0.25'', not ``.25''. Use ``cm\textsuperscript{3}'', not ``cc''.)
\end{itemize}

\subsection{Equations}
Number equations consecutively. To make your 
equations more compact, you may use the solidus (~/~), the exp function, or 
appropriate exponents. Italicize Roman symbols for quantities and variables, 
but not Greek symbols. Use a long dash rather than a hyphen for a minus 
sign. Punctuate equations with commas or periods when they are part of a 
sentence, as in:
\begin{equation}
a+b=\gamma\label{eq}
\end{equation}

Be sure that the 
symbols in your equation have been defined before or immediately following 
the equation. Use ``\eqref{eq}'', not ``Eq.~\eqref{eq}'' or ``equation \eqref{eq}'', except at 
the beginning of a sentence: ``Equation \eqref{eq} is . . .''

\subsection{\LaTeX-Specific Advice}

Please use ``soft'' (e.g., \verb|\eqref{Eq}|) cross references instead
of ``hard'' references (e.g., \verb|(1)|). That will make it possible
to combine sections, add equations, or change the order of figures or
citations without having to go through the file line by line.

Please don't use the \verb|{eqnarray}| equation environment. Use
\verb|{align}| or \verb|{IEEEeqnarray}| instead. The \verb|{eqnarray}|
environment leaves unsightly spaces around relation symbols.

Please note that the \verb|{subequations}| environment in {\LaTeX}
will increment the main equation counter even when there are no
equation numbers displayed. If you forget that, you might write an
article in which the equation numbers skip from (17) to (20), causing
the copy editors to wonder if you've discovered a new method of
counting.

{\BibTeX} does not work by magic. It doesn't get the bibliographic
data from thin air but from .bib files. If you use {\BibTeX} to produce a
bibliography you must send the .bib files. 

{\LaTeX} can't read your mind. If you assign the same label to a
subsubsection and a table, you might find that Table I has been cross
referenced as Table IV-B3. 

{\LaTeX} does not have precognitive abilities. If you put a
\verb|\label| command before the command that updates the counter it's
supposed to be using, the label will pick up the last counter to be
cross referenced instead. In particular, a \verb|\label| command
should not go before the caption of a figure or a table.

Do not use \verb|\nonumber| inside the \verb|{array}| environment. It
will not stop equation numbers inside \verb|{array}| (there won't be
any anyway) and it might stop a wanted equation number in the
surrounding equation.

\subsection{Some Common Mistakes}\label{SCM}
\begin{itemize}
\item The word ``data'' is plural, not singular.
\item The subscript for the permeability of vacuum $\mu_{0}$, and other common scientific constants, is zero with subscript formatting, not a lowercase letter ``o''.
\item In American English, commas, semicolons, periods, question and exclamation marks are located within quotation marks only when a complete thought or name is cited, such as a title or full quotation. When quotation marks are used, instead of a bold or italic typeface, to highlight a word or phrase, punctuation should appear outside of the quotation marks. A parenthetical phrase or statement at the end of a sentence is punctuated outside of the closing parenthesis (like this). (A parenthetical sentence is punctuated within the parentheses.)
\item A graph within a graph is an ``inset'', not an ``insert''. The word alternatively is preferred to the word ``alternately'' (unless you really mean something that alternates).
\item Do not use the word ``essentially'' to mean ``approximately'' or ``effectively''.
\item In your paper title, if the words ``that uses'' can accurately replace the word ``using'', capitalize the ``u''; if not, keep using lower-cased.
\item Be aware of the different meanings of the homophones ``affect'' and ``effect'', ``complement'' and ``compliment'', ``discreet'' and ``discrete'', ``principal'' and ``principle''.
\item Do not confuse ``imply'' and ``infer''.
\item The prefix ``non'' is not a word; it should be joined to the word it modifies, usually without a hyphen.
\item There is no period after the ``et'' in the Latin abbreviation ``et al.''.
\item The abbreviation ``i.e.'' means ``that is'', and the abbreviation ``e.g.'' means ``for example''.
\end{itemize}
An excellent style manual for science writers is \cite{b7}.

\subsection{Authors and Affiliations}
\textbf{The class file is designed for, but not limited to, six authors.} A 
minimum of one author is required for all conference articles. Author names 
should be listed starting from left to right and then moving down to the 
next line. This is the author sequence that will be used in future citations 
and by indexing services. Names should not be listed in columns nor group by 
affiliation. Please keep your affiliations as succinct as possible (for 
example, do not differentiate among departments of the same organization).

\subsection{Identify the Headings}
Headings, or heads, are organizational devices that guide the reader through 
your paper. There are two types: component heads and text heads.

Component heads identify the different components of your paper and are not 
topically subordinate to each other. Examples include Acknowledgments and 
References and, for these, the correct style to use is ``Heading 5''. Use 
``figure caption'' for your Figure captions, and ``table head'' for your 
table title. Run-in heads, such as ``Abstract'', will require you to apply a 
style (in this case, italic) in addition to the style provided by the drop 
down menu to differentiate the head from the text.

Text heads organize the topics on a relational, hierarchical basis. For 
example, the paper title is the primary text head because all subsequent 
material relates and elaborates on this one topic. If there are two or more 
sub-topics, the next level head (uppercase Roman numerals) should be used 
and, conversely, if there are not at least two sub-topics, then no subheads 
should be introduced.

\subsection{Figures and Tables}
\paragraph{Positioning Figures and Tables} Place figures and tables at the top and 
bottom of columns. Avoid placing them in the middle of columns. Large 
figures and tables may span across both columns. Figure captions should be 
below the figures; table heads should appear above the tables. Insert 
figures and tables after they are cited in the text. Use the abbreviation 
``Fig.~\ref{fig}'', even at the beginning of a sentence.

\begin{table}[htbp]
\caption{Table Type Styles}
\begin{center}
\begin{tabular}{|c|c|c|c|}
\hline
\textbf{Table}&\multicolumn{3}{|c|}{\textbf{Table Column Head}} \\
\cline{2-4} 
\textbf{Head} & \textbf{\textit{Table column subhead}}& \textbf{\textit{Subhead}}& \textbf{\textit{Subhead}} \\
\hline
copy& More table copy$^{\mathrm{a}}$& &  \\
\hline
\multicolumn{4}{l}{$^{\mathrm{a}}$Sample of a Table footnote.}
\end{tabular}
\label{tab1}
\end{center}
\end{table}

\begin{figure}[htbp]
\centerline{\includegraphics{fig1.png}}
\caption{Example of a figure caption.}
\label{fig}
\end{figure}

Figure Labels: Use 8 point Times New Roman for Figure labels. Use words 
rather than symbols or abbreviations when writing Figure axis labels to 
avoid confusing the reader. As an example, write the quantity 
``Magnetization'', or ``Magnetization, M'', not just ``M''. If including 
units in the label, present them within parentheses. Do not label axes only 
with units. In the example, write ``Magnetization (A/m)'' or ``Magnetization 
\{A[m(1)]\}'', not just ``A/m''. Do not label axes with a ratio of 
quantities and units. For example, write ``Temperature (K)'', not 
``Temperature/K''.

\section*{Acknowledgment}

The preferred spelling of the word ``acknowledgment'' in America is without 
an ``e'' after the ``g''. Avoid the stilted expression ``one of us (R. B. 
G.) thanks $\ldots$''. Instead, try ``R. B. G. thanks$\ldots$''. Put sponsor 
acknowledgments in the unnumbered footnote on the first page.

\section*{References}

Please number citations consecutively within brackets \cite{b1}. The 
sentence punctuation follows the bracket \cite{b2}. Refer simply to the reference 
number, as in \cite{b3}---do not use ``Ref. \cite{b3}'' or ``reference \cite{b3}'' except at 
the beginning of a sentence: ``Reference \cite{b3} was the first $\ldots$''

Number footnotes separately in superscripts. Place the actual footnote at 
the bottom of the column in which it was cited. Do not put footnotes in the 
abstract or reference list. Use letters for table footnotes.

Unless there are six authors or more give all authors' names; do not use 
``et al.''. Papers that have not been published, even if they have been 
submitted for publication, should be cited as ``unpublished'' \cite{b4}. Papers 
that have been accepted for publication should be cited as ``in press'' \cite{b5}. 
Capitalize only the first word in a paper title, except for proper nouns and 
element symbols.

For papers published in translation journals, please give the English 
citation first, followed by the original foreign-language citation \cite{b6}.

\begin{thebibliography}{00}
\bibitem{b1} KUSHLEV, Kostadin; PROULX, Jason DE; DUNN, Elizabeth W. Digitally connected, socially disconnected: The effects of relying on technology rather than other people. Computers in Human Behavior, v. 76, p. 68-74, 2017.
\bibitem{b2} KHALILI-MAHANI, Najmeh; SMYRNOVA, Anna; KAKINAMI, Lisa. To each stress its own screen: a cross-sectional survey of the patterns of stress and various screen uses in relation to self-admitted screen addiction. Journal of medical Internet research, v. 21, n. 4, p. e11485, 2019.
\bibitem{b3} ELHAI, Jon D. et al. Depression, anxiety and fear of missing out as correlates of social, non-social and problematic smartphone use. Addictive behaviors, v. 105, p. 106335, 2020.
\bibitem{b4} WANG, Jin-Liang; SHENG, Jia-Rong; WANG, Hai-Zhen. The association between mobile game addiction and depression, social anxiety, and loneliness. Frontiers in public health, v. 7, p. 247, 2019.
\bibitem{b5} RODRÍGUEZ-GARCÍA, Antonio-Manuel; MORENO-GUERRERO, Antonio-José; LOPEZ BELMONTE, Jesus. Nomophobia: An individual’s growing fear of being without a smartphone—a systematic literature review. International journal of environmental research and public health, v. 17, n. 2, p. 580, 2020.
\bibitem{b6} ZARA, M. C.; MONTEIRO, L. H. A. The negative impact of technological advancements on mental health: An epidemiological approach. Applied Mathematics and Computation, v. 396, p. 125905, 2021.
\bibitem{b7} LEE, MyungSuk; HAN, MuMoungCho; PAK, JuGeon. Analysis of behavioral characteristics of smartphone addiction using data mining. Applied Sciences, v. 8, n. 7, p. 1191, 2018.
\bibitem{b8} ZHANG, Xingxu et al. Correlation Analysis of Internet Addiction with Daily Behavior: A Data-Driven Method. In: Proceedings of the 2020 3rd International Conference on Big Data Technologies. 2020. p. 41-46.
\bibitem{b9} KIM, Eun Yeob; KIM, Seok Hwan. Pattern Study on smartphone addiction group. Journal of the Korea Academia-Industrial cooperation Society, v. 16, n. 2, p. 1207-1215, 2015.
\bibitem{b10} RYU, Myeong-Un et al. A Smartphone Addiction and Disease Prevention System Through the Collection and Analysis of Smartphone Usage Patterns. Journal of Internet Computing and Services, v. 16, n. 3, p. 95-104, 2015.
\bibitem{b11} PURWANDARI, Betty et al. Internet Addiction and Mental Health Prediction Using Ensemble Learning Based on Web Browsing History. In: Proceedings of the 3rd International Conference on Software Engineering and Information Management. 2020. p. 155-159.

\end{thebibliography}
\vspace{12pt}
\color{red}
IEEE conference templates contain guidance text for composing and formatting conference papers. Please ensure that all template text is removed from your conference paper prior to submission to the conference. Failure to remove the template text from your paper may result in your paper not being published.

\end{document}
